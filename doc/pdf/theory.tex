\documentclass[a4paper,american,floatfix,pdftex,superscriptaddress,twoside,%
aps,pra,
% citeautoscript,% leave away for latexdiff
linenumbers,% remove for arXiv
% longbibliography,% not needed for revtex 4.2
% preprint,%
% galley,%
reprint,%final% add final to see real layout, no todonotes anymore, ...
]{revtex4-2}%
\usepackage{amsfonts,amsmath,amssymb}
\usepackage[T1]{fontenc}
\usepackage{graphicx}%
\usepackage[utf8]{inputenc}
\usepackage{mathtools}
\usepackage{hyperref, hypernat}
\usepackage[displaymath,textmath,graphics]{preview}
\DeclarePairedDelimiter{\ceil}{\lceil}{\rceil}
\graphicspath{{./figures/}} % define figure directory

\newcommand{\cfeldesy}{\affiliation{Center for Free-Electron Laser Science, Deutsches
      Elektronen-Synchrotron DESY, Notkestraße 85, 22607 Hamburg, Germany}}%
\newcommand{\ezemail}{\email{emil.zak@cfel.de}}%


\begin{document}
\title{CHIRALEX: theory}

\author{Emil J.\ Zak}\ezemail\cfeldesy% 
\date{\today}%
\begin{abstract}\noindent%


\end{abstract}
\maketitle%

\section{Introduction}

Essential parameters are kept in the \textit{params} dictionary, which is passed throughout the program. It is used to construct all types of objects, from grids to the propagator object.

\section{basis functions and grids}

\subsection{Radial quadratures}

\subsection{Radial grid}
\label{sec:radgrid}
The radial grid used to solve the TDSE is based on an underlying Gauss-Lobatto (GL) quadrature scheme. 
First, a generic quadrature grid is created: $\lbrace x_k\rbrace_{k=0,1,...,Nlobs-1} \in [-1,1]$. Its size is determined by the \textit{Nlobs} input parameter ($Nl$ for short notation). Note that the boundary points in the Gauss-Lobatto grid are $-1$ and $+1$. 
The GL grid is then scaled to reflect the bin size $R_b$ requested by the user ($binw$ input parameter):

\begin{equation}
S_k = \frac{1}{2}R_b\left(x_k + 1\right)
\end{equation} 

Bin size must be determined empirically to ensure good convergence and stability of the results.
Next, the scaled grid is copied and translated by the bin size to generate the full primitive radial grid $G_{prim}$:

\begin{equation}
r_{ik} = S_k + T_i, \qquad i=0,1,...,Nb-1 \;,k=0,1,...,Nlobs-1 
\end{equation} 
where $T_i = i \cdot R_b + \epsilon$. The constant shift $\epsilon$ (rshift keyword) is optional and by default is set to 0.
The resulting grid has the size $Nb \times Nlobs $ and contains boundary points $r_{00} = \epsilon$ and $r_{Nb-1,Nlobs-1} = R_{max}$. The last point from bin $i$ and the first point from bin $i+1$ are identical in the primitive grid.

The \textit{coupled grid} $G$ is generated by removing boundary points, plus merging duplicate points at bin boundaries in $G_{prim}$. Through these operations the size of the coupled grid is $Nr =  Nb \times (Nlobs - 1) -1 $. Note that all indices start from $0$ for python compatibility. For example $G_{00} = r_{01}$ and $G_{0,Nl-3} = r_{0,Nl-2}$. The boundary of the first bin is at $G_{0,Nl-2} = r_{0,Nl-1} = r_{10}$. The last point in the coupled grid is $G_{Nb-1,Nl-3} = r_{Nb-1,Nl-2}$.


\subsection{Radial basis functions}

\subsubsection{Interpolated radial functions}

\subsection{Angular basis functions}


\section{index mapping}
A natural choice for mapping the basis set indices is to set the grid points as major dimension (changing last). Such a choice offers a simple way of cutting the Hamiltonian matrix at a given grid point by slicing the basis. We name this convention as 'DVR' mapping. Bridge functions, as discussed in \ref{sec:radgrid}, are defined for indices $n = Nl-2$, that is the last index in each bin. A consequence of such a choice is a near block-diagonal structure of the Hamiltonian. Another common choice is to choose bridge functions as first in the bin ($n = 0$), which results in 'arms' reaching out from bin $i$ into bin $i+1$, as shown in \ref{fig:ham_struct_comp}. Another option given in the code is to select mapping in which the angular quantum numbers $l,m$ are the major dimension. We name this convention 'SPECTRAL'. The choice for mapping is determined by the 'map\_type' = 'DVR,SPECT' keyword. With the 'DVR' convention the map is given as follows:

\begin{equation}
p(i,n,l,m) = \left(i\cdot (Nl-1) + n\right) \cdot \left( l_{max}+1\right)^2 +l\cdot(l+1) + m
\end{equation} 
with $p(0,0,0,0) =0 $. For $l_{max} = 0$ we have $p(Nb-1,Nl-3,0,0) =  Nb \times (Nl - 1) -2 $, which retrieves the size of the coupled radial grid/basis set. The map is stored in a numpy array ('maparray') and is saved in file. The ranges for the radial indices $i$ and $n$ are the same as for the coupled grid, i.e. $i=0,1,2,...Nb-1$ and $n=0,1,2,...,Nl-2$ for $i<Nb-1$ and $n=0,1,2,...,Nl-3$ for $i=Nb-1$.
\begin{table}[h!]
	\begin{center}
		\caption{Example 'DVR' coupled basis set index mapping for $Nb=3$, $Nl=3$, $lmax=1$.}
		\label{tab:mapping}
		\begin{tabular}{l|l|l|l|r|l}
			\textbf{i} & \textbf{n} & \textbf{$\xi$} & \textbf{l} &\textbf{m} & \textbf{p}\\
			\hline
			0 & 0 & 0 & 0 &0 & 0\\
			0 & 0 & 0 & 1 &-1 & 1\\
			0 & 0 & 0 & 1 &0 & 2\\
			0 & 0 & 0 & 1 &+1 & 3\\
			0 & 1 & 1 & 0 &0 & 4\\
			0 & 1 & 1 & 1 &-1 & 5\\
			0 & 1 & 1 & 1 &0 & 6\\
			0 & 1 & 1 & 1 &+1 & 7\\
			...\\
			2 & 0 & 4 & 1 &0 & 18\\
			2 & 0 & 4 & 1 &+1 & 19\\		
		\end{tabular}
	\end{center}
\end{table}


\section{Rotated electrostatic potential}
Each molecular-frame orientation gives different electrostatic potential as seen from the laboratory frame, in which the basis set is defined. This means that for each orientation one must generate generally a different Lebedev quadrature grid. Even if a global scheme is used, the values of the electrostatic potential will be different at these points for each orientation. This is because the lebedev grid is defined in the laboratory frame. Therefore at this stage, a separate Psi4 calculation must be initialized for each orientation. I do not see a way around it at the moment, other than performing multipole expansion and calculating the angular part (spherical tensor form) of the potential analytically. But finding this expansion is costly, even more than running Psi4 for every orientation.

\section{Kinetic energy operator}

\section{Potential energy operator}



\section{Hamiltonian}
The current paradigm assumes calculation of the sparse Hamiltonian matrix and feeding this matrix into an eigensolver or propagator routine. In future versions, for better computational performance it may be useful not to calculate the Hamiltonian matrix, but rather code the matrix-vector product only, as part of an interface routine with appropriate iterative eigensolver.
Due to high degree of sparsity of the Hamiltonians appearing in our calculations, it is sufficient to keep in memory the whole sparse matrix.

We decided to keep the bound state Hamiltonian $H_0$ and the initial propagation Hamiltonian $H_{init}$ as separate entities, calculated independently. The computational overhead related to calculating the bound-state part twice is marginal, as we assume that the full propagation basis is much bigger than the basis for the bound Hamiltonian $Nbas \gg Nbas0$. Future releases might recycle the bound state KEO and POT and build only the outside region Hamiltonian.

\section{Wavepacket propagation}
\subsection{time grid}
The time grid for the calculation is determined by the input keywords:
\begin{table}[h!]
	\begin{center}
		\caption{Time grid keywords.}
		\label{tab:time-grids}
		\begin{tabular}{l|c|r}
			\textbf{keyword} & \textbf{description} & \textbf{type}\\
			\hline
			t0 & start time & float\\
			tmax & end time & float\\
			dt & time step & float\\
			time\_units & units & string \\
			wfn\_saverate & save rate of the wavepacket  & float \\
			
		\end{tabular}
	\end{center}
\end{table}

The generated grid is equidistant given by the following formula:
\begin{equation}
t[i] = t_0 + i \cdot dt ;\ i=0,1,2,...,Ntpts
\end{equation}
where 
\begin{equation}
Ntpts = \ceil[\Bigg]{\frac{t_{max}-t_0}{dt}}
\end{equation}
such that the array of time-points contains elements: $t[0],t[1],...,t[Ntpts]$.
Note that the last point (tmax) is included in the grid. 

\section{Input file}


\begin{itemize}
	\item \textit{molec\_name}: name of the molecule (chiralium, h, h2s, etc.)
	\item \textit{matelem\_method}: method of choice for the calculation of the potential energy matrix elements. Availability of methods given below depends on the chosen molecule.
	\begin{itemize}
		\item \textit{analytic}: use analytic formulas for the matrix elements
		\item \textit{lebedev}: use own implementation of the spherical Lebedev quadratures with fixed global quadrature level for all grid points defined in \textit{sph\_quad\_global} = "lebedev\_119", "lebedev\_131", etc.
		\item \textit{lebedev\_adaptive}: use own implementation of the spherical Lebedev quadratures with adaptive quadrature level, optimized for each grid point. Stopping condition is based on convergence criteria defined in \textit{sph\_quad\_tol} tolerance (in a.u.) for the convergence of matrix elements with the maximal values of the $l$ quantum number. 
		\item \textit{quadpy}: use the Quadpy library to calculate the spherical integrals with Lebedev quadratures.
		\item \textit{others}: to be implemented.
	\end{itemize}

\end{itemize}
\bibliography{theory}
\end{document}

